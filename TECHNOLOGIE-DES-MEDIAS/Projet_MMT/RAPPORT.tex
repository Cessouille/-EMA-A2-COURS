\documentclass[12pt,a4paper]{article}
\usepackage[utf8]{inputenc}
\usepackage[french]{babel}
\usepackage[T1]{fontenc}
\usepackage{graphicx}
\usepackage{hyperref}
\usepackage{amsmath}
\usepackage{geometry}
\usepackage{float}
\geometry{margin=2.5cm}

\title{Rapport MMT}
\author{Célian CHAUSSON}
\date{Octobre 2025}

\begin{document}

\maketitle

\section{Analyse du PCM}

\subsection*{a. Création d'une tonalité sinusoïdale}
Créer une tonalité sinusoïdale de fréquence $f = 2$ kHz, de 3 secondes de durée, en utilisant 10 échantillons (en float) par période. Reproduire cette tonalité sur les haut-parleurs de votre ordinateur.

\subsection*{b. Quantification à 8 bits/échantillon}
Quantifier ce signal en (int) à 8 bits/ech.

Écouter fichier pcm\_8bits.wav

\subsection*{c. Quantification avec différentes résolutions}
Quantifier ce signal en utilisant une résolution de 6 bits/ech, de 4 bits/ech, de 3 bits/ech et de 2 bits/ech.

\begin{itemize}
    \item Écouter fichier pcm\_6bits.wav
    \item Écouter fichier pcm\_4bits.wav
    \item Écouter fichier pcm\_3bits.wav
    \item Écouter fichier pcm\_2bits.wav
\end{itemize}

\subsection*{d. Quantification à 1 bit/échantillon}
Que se passe-t-il quand la résolution du quantificateur devient 1 bit/ech ?

Écouter fichier pcm\_1bits.wav

\textbf{Analyse :} À 1 bit/échantillon, le quantificateur ne conserve que le signe du signal (positif ou négatif). Le signal reconstruit est une onde carrée alternant entre +1 et -1. Le son devient très mauvais, limite impossible à écouter, car seule la fréquence fondamentale est partiellement préservée avec de nombreuses harmoniques parasites.

\textbf{Graphiques de comparaison :}

\begin{figure}[H]
    \centering
    \includegraphics[width=0.8\textwidth]{PCM/pcm_comparison.png}
    \caption{Comparaisons individuelles pour chaque résolution}
    \label{fig:pcm_comparison}
\end{figure}

\begin{figure}[H]
    \centering
    \includegraphics[width=0.8\textwidth]{PCM/pcm_all_comparison.png}
    \caption{Superposition de tous les niveaux}
    \label{fig:pcm_all_comparison}
\end{figure}

\section{Analyse du DPCM}

\subsection*{a. Comment se porte ce codeur en présence d'erreurs aléatoires}
Comment se porte ce codeur si on est en présence d'erreurs aléatoires avec un taux d'erreur $p = 10^{-2}$ et $p = 10^{-3}$. Conclusions ?

\textbf{Configuration :} Signal sinusoïdal 2 kHz, résolution $R = 8$ bits

\begin{itemize}
    \item Écouter fichier dpcm\_R8\_errors\_p1e\_02.wav
    \item Écouter fichier dpcm\_R8\_errors\_p1e\_03.wav
\end{itemize}

\textbf{Analyse des résultats :}

\begin{itemize}
    \item \textbf{Avec $p = 10^{-2}$ :} Des erreurs sonores apparaissent. Le signal reste reconnaissable mais la qualité est dégradée de manière notable.
    
    \item \textbf{Avec $p = 10^{-3}$ :} La qualité reste correcte. Les erreurs sont moins perceptibles et espacées. Le signal est proche de la qualité sans erreur.
\end{itemize}

\begin{figure}[H]
    \centering
    \includegraphics[width=0.9\textwidth]{DPCM/dpcm_comparison.png}
    \caption{Comparaison DPCM - (1) Signal original, (2) DPCM avec $p = 10^{-2}$, (3) DPCM avec $p = 10^{-3}$}
    \label{fig:dpcm_comparison}
\end{figure}

\textbf{Conclusions :}

\begin{itemize}
    \item Le DPCM est sensible aux erreurs binaires car chaque échantillon décodé dépend du précédent
    \item Notre implémentation utilise le signal original pour éviter l'accumulation catastrophique des erreurs (réinitialisation du prédicteur)
    \item Comparé au PCM, le DPCM nécessite des mécanismes de protection plus robustes (codes correcteurs, réinitialisation périodique)
    \item Pour $p < 10^{-3}$, le système reste acceptable ; au-delà de $10^{-2}$, la qualité se dégrade rapidement
\end{itemize}

\subsection*{b. Quantification de la voix avec DPCM}
Quantifier la voix de Xtine en utilisant une résolution de 8 bits/ech. Que se passe-t-il si on a un taux d'erreur binaire $p = 10^{-2}$ ?

\textbf{Configuration :} Signal vocal ``XTINE'', résolution $R = 8$ bits, taux d'erreur $p = 10^{-2}$

Écouter fichier xtine\_dpcm\_R8\_errors\_p1e\_02.wav

\textbf{Analyse :}

\begin{itemize}
    \item \textbf{Signal sans erreur :} Le DPCM à 8 bits reconstruit le signal vocal avec une qualité élevée. Le SNR est élevé et la parole reste naturelle.
    
    \item \textbf{Avec erreurs $p = 10^{-2}$ :}
    \begin{itemize}
        \item La qualité vocale reste intelligible
        \item Présence de clics ou artefacts audibles aux points d'erreur
        \item La voix conserve son timbre général
        \item Les erreurs sont perceptibles mais n'empêchent pas la compréhension
    \end{itemize}
\end{itemize}

\begin{figure}[H]
    \centering
    \includegraphics[width=0.9\textwidth]{XTINE/xtine_dpcm_comparison.png}
    \caption{Comparaison signal vocal XTINE - (1) Signal original, (2) DPCM $R=8$ sans erreurs, (3) DPCM $R=8$ avec erreurs $p = 10^{-2}$}
    \label{fig:xtine_dpcm_comparison}
\end{figure}

\textbf{Comparaison avec le signal sinusoïdal :}

\begin{itemize}
    \item Le signal vocal est plus complexe (non périodique, large bande spectrale)
    \item Les erreurs sont potentiellement moins perceptibles car masquées par le contenu vocal
    \item L'oreille humaine est plus tolérante aux artefacts dans la parole que dans les tons purs
\end{itemize}

\textbf{Conclusion :} Le DPCM à 8 bits est adapté pour la compression de la parole, mais un taux d'erreur de $10^{-2}$ introduit des dégradations audibles. Pour des applications critiques (téléphonie, streaming), un taux d'erreur $< 10^{-3}$ est recommandé, avec des mécanismes de protection (codes correcteurs d'erreurs, réinitialisation périodique du prédicteur).

\end{document}
